\chapter{More Voting Systems}

In the previous chapter we looked at a number of voting systems that are in widespread use in various democracies. Here we look at three voting systems that are not used for mass elections anywhere around the world, though all of them have been used in various purposes for combining the views of groups. (For instance, they have been used for elections in small groups.)

\section{Borda Count}
In a Borda Count election, each voter ranks each of the candidates, as in Instant Runoff Voting. Each candidate then receives $n$ points for each first place vote they receive (where $n$ is the number of candidates), $n-1$ points for each second place vote, and so on through the last place candidate getting 1 point. The candidate with the most points wins.

One nice advantage of the Borda Count is that it eliminates the chance for the kind of strategic voting that exists in Instant Runoff Voting, or for that matter any kind of Runoff Voting. It can never make it more likely that $A$ will win by someone changing their vote away from $A$. Indeed, this could only lead to $A$ having fewer votes. This certainly seems to be reasonable.

Another advantage is that many preferences beyond first place votes count. A candidate who is every single voter's second best choice will not do very well under any voting system that gives a special weight to first preferences. But such a candidate may well be in a certain sense the best representative of the whole community.

And a third advantage is that the Borda Count includes a rough approximation of voter's strength of preference. If one voter ranks $A$ a little above $B$, and another votes $B$ many places above $A$, that's arguably a sign that $B$ is a better representative of the two of them than $A$. Although only one of the two prefers $B$, one voter will be a little disappointed that $B$ wins, while the other would be very disappointed if $B$ lost.

These are not trivial advantages. But there are also many disadvantages which explain why no major electoral system has adopted Borda Count voting yet, despite its strong support from some theorists.

First, Borda Count is particularly complicated to implement. It is just as difficult for the voter to as in Instant Runoff Voting; in each case they have to express a complete preference ordering. But it is much harder to count, because the vote counter has to detect quite a bit of information from each ballot. Getting this information from millions of ballots is not a trivial exercise.

Second, Borda Count has a serious problem with `clone candidates'. In plurality voting, a candidate suffers if there is another candidate much like them on the ballot. In Borda Count, a candidate can seriously gain if such a candidate is added. Consider the following situation. In a certain electorate, of say 100,000 voters, 60\% of the voters are Republicans, and 40\% are Democrats. But there is only one Republican, call them $R$, on the ballot, and there are 2 Democrats, $D1$ and $D2$ on the ballot. Moreover, $D2$ is clearly a worse candidate than $D1$, but the Democrats still prefer the Democrat to the Republican. Since the district is overwhelmingly Republican, intuitively the Republican should win. But let's work through what happens if 60,000 Republicans vote for $R$, then $D1$, then $D2$, and the 40,000 Democrats vote $D1$ then $D2$ then $R$. In that case, $R$ will get $60,000 \times 3 + 40,000 \times 1 = 220,000$ points, $D1$ will get $60,000 \times 2 + 40,000 \times 3 = 240,000$ points, and $D2$ will get $60,000 \times 1 + 40,000 \times 2 = 140,000$ points, and $D1$ will win. Having a `clone' on the ticket was enough to push $D1$ over the top.

On the one hand, this may look a lot like the mirror image of the `spoiler' problem for plurality voting. But in another respect it is much worse. It is hard to get someone who is a lot ideologically like your opponent to run in order to improve your electoral chances. It is much easier to convince someone who already wants you to win to add their name to the ballot in order to improve your chances. In practice, this would either lead to an arms race between the two parties, each trying to get the most names onto the ballot, or very restrictive (and hence undemocratic) rules about who was even allowed to be on the ballot, or, most likely, both.

The third problem comes from thinking through the previous problem from the point of view of a Republican voter. If the Republican voters realise what is up, they might vote tactically for $D2$ over $D1$, putting $R$ back on top. In a case where the electorate is as partisan as in this case, this might just work. But this means that Borda Count is just as susceptible to tactical voting as other systems; it is just that the tactical voting often occurs downticket. (There are more complicated problems, that we won't work through, about what happens if the voters mistakenly judge what is likely to happen in the election, and tactical voting backfires.)

Finally, it's worth thinking about whether the supposed major virtue of Borda Count, the fact that it considers all preferences and not just first choices, is a real gain. The core idea behind Borda Count is that all preferences should count equally. So the difference between first place and second place in a voter's affections counts just as much as the difference between third and fourth. But for many elections, this isn't how the voters themselves feel. I suspect many people reading this have strong feelings about who was the best candidate in the past Presidential election. I suspect very few people had strong feelings about who was the third best versus fourth best candidate. This is hardly a coincidence; people identify with a party that is their first choice. They say, ``I'm a Democrat'' or ``I'm a Green'' or ``I'm a Republican''. They don't identify with their third versus fourth preference. Perhaps voting systems that give primary weight to first place preferences are genuinely reflecting the desires of the voters. 

\section{Approval Voting}
In plurality voting, every voter gets to vote for one candidate, and the candidate with the most votes wins. Approval voting is similar, except that each voter is allowed to vote for as many candidates as they like. The votes are then added up, and the candidate with the most votes wins. Of course, the voter has an interest in not voting for too many candidates. If they vote for all of the candidates, this won't advantage any candidate; they may as well have voted for no candidates at all.

The voters who are best served by approval voting, at least compared to plurality voting, are those voters who wish to vote for a non-major candidate, but who also have a preference between the two major candidates. Under approval voting, they can vote for the minor candidate that they most favor, and also vote for the the major candidate who they hope will win. Of course, runoff voting (and Instant Runoff Voting) also allow these voters to express a similar preference. Indeed, the runoff systems allow the voters to express not only two preferences, but express the order in which they hold those preferences. Under approval voting, the voter only gets to vote for more than one candidate, they don't get to express any ranking of those candidates.

But arguably approval voting is easier on the voter. The voter can use a ballot that looks just like the ballot used in plurality voting. And they don't have to learn about preference flows, or Borda Counts, to understand what is going on in the voting. Currently there are many voters who vote for, or at least appear to try to vote for, multiple candidates. This is presumably inadvertent, but approval voting would let these votes be counted, which would refranchise a number of voters. Approval voting has never been used as a mass electoral tool, so it is hard to know how quick it would be to count, but presumably it would not be incredibly difficult.\footnote{There is one circumstance when something like Approval Voting is used in the United States. It is not uncommon when there are $n > 1$ candidates to be elected to a position, like school board members, that each voter can select $n$ candidates, and the top $n$ vote getters are elected. This isn't true Approval Voting, since there is still a cap on how many people you can vote for.}

One striking thing about approval voting is that it is not a function from voter preferences to group preferences. Hence it is not subject to the Arrow Impossibility Theorem. It isn't such a function because the voters have to not only rank the candidates, they have to decide where on their ranking they will `draw the line' between candidates that they will vote for, and candidates that they will not vote for. Consider the following two sets of voters. In each case candidates are listed from first preference to last preference, with stars indicating which candidates the voters vote for.

\starttab{c c c p{100pt} c c c}
40\% & 35\% & 25\% & & 40\% & 35\% & 25\% \\
\cmidrule(r){1-3}
\cmidrule(r){5-7}
$*A$ & $*B$ & $*C$ & & $*A$ & $*B$ & $*C$ \\
$B$ & $A$ & $B$ & & $B$ & $A$ & $*B$ \\
$C$ & $C$ & $A$ & & $C$ & $C$ & $A$
\stoptab In the election on the left-hand-side, no voter takes advantage of approval voting to vote for more than one candidate. So $A$ wins with 40\% of the vote. In the election on the right-hand-side, no one's preferences change. But the 25\% who prefer $C$ also decide to vote for $B$. So now $B$ has 60\% of the voters voting for them, as compared to 40\% for $A$ and 25\% for $C$, so $B$ wins.

This means that the voting system is not a function from voter preferences to group preferences. If it were a function, fixing the group preferences would fix who wins. But in this case, without a single voter changing their preference ordering of the candidates, a different candidate won. Since the Arrow Impossibility Theorem only applies to functions from voter preferences to group preferences, it does not apply to Approval Voting.

\section{Range Voting}
In Range Voting, every voter gives each candidate a score. Let's say that score is from 0 to 10. The name `Range' comes from the range of options the voter has. In the vote count, the score that each candidate receives from each voter is added up, and the candidate with the most points wins.

In principle, this is a way for voters to express very detailed opinions about each of the candidates. They don't merely rank the candidates, they measure how much better each candidate is than all the other candidates. And this information is then used to form an overall ranking of the various candidates.

In practice, it isn't so clear this would be effective. Imagine that a voter $V$ thinks that candidate $A$ would be reasonably good, and candidate $B$ would be merely OK, and that no other candidates have a serious chance of winning. If $V$ was genuinely expressing their opinions, they might think that $A$ deserves an 8 out of 10, and $B$ deserves a 5 out of 10. But $V$ wants $A$ to win, since $V$ thinks $A$ is the better candidate. And $V$ knows that what will make the biggest improvement in $A$'s chances is if they score $A$ a 10 out of 10, and $B$ a 0 out of 10. That will give $A$ a 10 point advantage, whereas they may only get a 3 point advantage if the voter voted sincerely.

It isn't unusual for a voter to find themselves in $V$'s position. So we might suspect that although Range Voting will give the voters quite a lot of flexibility, and give them the chance to express detailed opinions, it isn't clear how often it would be in a voter's interests to use these options.

And Range Voting is quite complex, both from the perspective of the voter and of the vote counter. There is a lot of information to be gleaned from each ballot in Range Voting. This means the voter has to go to a lot of work to fill out the ballot, and the vote counter has to do a lot of work to process all that information. This means that Range Voting might be very slow, both in terms of voting and counting. And if voters have a tactical reason for not wanting to fill in detailed ballots, this might mean it's a lot of effort for not a lot of reward, and that we should stick to somewhat simpler vote counting methods.
