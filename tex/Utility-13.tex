\section{Utility and Welfare}
So far we've frequently talked about the utility of various outcomes. What we haven't said a lot about is just what it is that we're measuring when we measure the utility of an outcomes. The intuitive idea is that utility is a measure of welfare - having outcomes with higher utility if a matter of having a higher level of welfare. But this doesn't necessarily move the idea forward, because we'd like to know a bit more about what it is to have more welfare. There are a number of ways we can frame the same question. We can talk about `well-being' instead of welfare, or we can talk about having a good life instead, or having a life that goes well. But the underlying philosophical question, what makes it the case that a life has these features, remains more or less the same.

There are three primary kinds of theories of welfare in contemporary philosophy. These are
\begin{itemize*}
\item Experience Based theories
\item Objective List theories
\item Preference Based theories
\end{itemize*}
\noindent In decision theory, and indeed in economics, people usually focus on preference based theories. Indeed, the term `utility' is sometimes used in such way that $A$ has more utility than $B$ just means that the agent prefers $A$ to $B$. Indeed, I've sometimes earlier moved back and forth previously between saying $A$ has higher utility and saying $A$ is preferred. And the focus here (and in the next set of notes) will be on why people have moved to preference based accounts, and technical challenges within those accounts. But we'll start with the non-preference based accounts of welfare.

\section{Experiences and Welfare}
One tradition, tracing back at least to Jeremy Bentham, is to identify welfare with having good experiences. A person's welfare is high if they have lots of pleasures, and few pains. More generally, a person's welfare is high if they have good experiences. 

Of course it is possible that a person might be increasing their welfare by having bad experiences at any one time. They might be at work earning the money they need to finance activities that lead to good experiences later, or they might just be looking for money to stave off bad experiences (starvation, lack of shelter) later. Or perhaps the bad experiences, such as in strenuous exercise, are needed in order to be capable of later doing the things, e.g. engaging in sporting activities, that produce good experiences. Either way, the point has to be that a person's welfare is not simply measured by what their experiences are like right now, but by what their experiences have been, are, and will be over the course of their lives.

There is one well known objection to any such account - what Robert Nozick called the ``experience machine''. Imagine that a person is, in their sleep, kidnapped and wired up to a machine that produces in their brain the experiences as of a fairly good life. The person still seems to be having good days filled with enjoyable experiences. And they aren't merely raw pleasurable sensations - the person is having experiences as of having rich fulfilling relationships with the friends and family they have known and loved for years. But in fact the person is not in any contact with those people, and for all the friends and family know, the person was kidnapped and killed. This continues for decades, until the person has a peaceful death at an advanced age.

Has this person had a good life or a bad life? Many people think intuitively that they have had a bad life. Their entire world has been based on an illusion. They haven't really had fulfilling relationships, travelled to exciting places, and so on. Instead they have been systematically deceived about the world. But on an experience based view of welfare, they have had all of the goods you could want in life. Their experiences are just the experiences that a person having a good life would have. So the experience based theorist is forced to say that they have had a good life, and this seems mistaken.

Many philosophers find this a compelling objection to the experience based view of welfare. But many people are not persuaded. So it's worth thinking a little through some other puzzles for purely experience based views of welfare.

It's easy enough to think about paradigmatic pains, or bad experiences. It isn't too hard to come up with paradigmatic good experiences, though perhaps there would be more disagreement about what experiences are paradigms of the good than are paradigms of the bad. But many experiences are less easy to classify. Even simple experiences like tickles might be good experiences for some, and bad experiences for others.

When we get to more complicated experiences, things are even more awkward for the experience based theorist. Some people like listening to heavily distorted music, or watching horror movies, or drinking pineapple schnapps. Other people, indeed most people, do not enjoy these things. The experience theory has a couple of choices here. Either we can say that one group is wrong, and these things either do, or do not, raise one's welfare. But this seems implausible for all experiences. Perhaps at the fringes there are experiences people seek that nevertheless decrease their welfare, but it seems strange to argue that the same experiences are good for everyone. 

The other option is to say that there are really two experiences going on when you, say, listen to a kind of music that some, but not all, people like. There is a `first-order' experience of hearing the music. And there is a `second-order' experience, an experience of enjoying the experience of hearing the music. Perhaps this is right in some cases. (Perhaps for horror movies, fans both feel horrified and have a pleasant reaction to being horrified, at least some of the time.) But it seems wrong in general. If there is a food that I like and you dislike, that won't usually be because I'll have a positive second-order experience, and you won't have such a thing. Intuitively, the experience of, say, drinking a good beer, isn't like that, because it just isn't that complicated. Rather, I just have a certain kind of experience, and I like it, and you, perhaps, do not.

A similar problem arises when considering the choices people make about how to distribute pleasures over their lifetime. Some people are prepared to undergo quite unpleasant experiences, e.g. working in painful conditions, in exchange for pleasant experiences later (e.g. early retirement, higher pay, shorter hours). Other people are not. Perhaps in some cases people are making a bad choice, and their welfare would be higher if they made different trade-offs. But this doesn't seem to be universally true - it just isn't clear that there's such a thing as the universally correct answer to how to trade off current unpleasantness for future pleasantness.

Note that this intertemporal trade-off question actually conceals two distinct questions we have to answer. One is how much we want to `discount' the future. Economists think, with some empirical support, that people mentally discount future goods. People value a dollar now more than they value a dollar ten years hence, or even an inflation adjusted dollar ten years hence. The same is true for experiences: people value good experiences now more than good experiences in the future. But it isn't clear how much discount, if any, is consistent with maximising welfare. The other question is how much we value high `peaks' of experience versus avoiding low `troughs'. Some people are prepared to put up with the bad to get the good, others are not. And the worry for the experience based theorist is that neither need be making a mistake. Perhaps what is best for a person isn't just a function of their experiences over time, but on how much they value the kind of experiences that they get.

So we've ended up with three major kinds of objections to experience based accounts of welfare.
\begin{itemize*}
\item The experience machine does not increase our welfare
\item Different people get welfare from different experiences
\item Different people get different amounts of welfare from the same sequences of experiences over time, even if they agree about the welfare of each of the moment-to-moment experiences.
\end{itemize*}
\noindent These seem like enough reasons to move to other theories of welfare.

\section{Objective List Theories}
One response to these problems with experience based accounts is to move to a theory based around desire satisfaction. Since that's the theory that's most commonly used in decision theory, we'll look at it last. Before that, we'll look briefly at so called \textit{objective list} theories of welfare. These theories hold that there isn't necessarily any one thing that makes your life better. Welfare isn't all about good experiences, or about having preferences that are satisfied. Rather, there are many ways in which your welfare can be improved. The list of things that make your life better may include:
\begin{itemize*}
\item Knowledge
\item Engaging in rational activity
\item Good health, adequate shelter, and more generally good physical well-being
\item Being in loving relationships, and in sustained friendships
\item Being virtuous
\item Experiencing beauty
\item Desiring the things that make life better, i.e. the things on this list
\end{itemize*}
\noindent Some objective list theorists hold that the things that should go on the list do have something in common, but this isn't an essential part of the theory.

The main attraction of the objective list approach is negative. We've already seen some of the problems with experience based theories of welfare. We'll see later some of the problems with desire based theories. A natural response to this is to think that welfare is heteroegenous, and that no simple theory of welfare can capture all that makes human lives go well. That's the response of the objective list theorist.

The first thing to note about these theories is that the lists in question always seem open to considerable debate. If there was a clearer principle about what's going on the lists and what is not, this would not be such a big deal. But in the absence of a clear (or easy to apply) principle, there is a sense of arbitrariness about the process.

Indeed, the lists that are produced by Western academics seem notably aligned with the desires and values of Western academics. It's notable that the lists produced tend to give very little role to the family, to religion, to community and to tradition. Of course all these things can come in indirectly. If being in loving relationships is a good, and families promote loving relationships, then families are an indirect good. And the same thing can be said religion, and community, and traditional practices. But still, many people might hold those things to be valuable in their own right, not just because of the goods that they produce. Or they might hold some things on the canonical lists, such as education and knowledge to be instrumental goods, rather than making them primary goods as philosophers often do.

This can't be an objection to objective list theories of welfare as such. Nothing in the theory rules out extending the list to include families, or traditions, in the mix, for instance. (Indeed, these kinds of goods are included in some versions of the theory.) But it is perhaps revealing that the lists hew so closely to the Western academic's idea of the good life. (Indeed the list I've got here is more universal than several proposed lists, since I've included health and shelter, which is left off some.) It might well be thought that there isn't one list of goods that make life good for any person in any community at any time. There might well be a list of what makes for a good life in a community like ours, and maybe even lists like the one above capture it, but claims to universality should be treated sceptically.

A more complicated question is how to generate comparative welfare judgments from the list. Utilities are meant to be represented numerically, so we need to be able to say which of two outcomes is better, or that the outcomes are exactly as good as one another. (Perhaps we need something more, some way of saying how much better one life is than another. But we'll set that question aside for now.) We already saw one hard aspect of this question above - how do we turn facts about the welfare of a person at different times of their life into an overall welfare judgment? That question is just as hard for the objective list theorist as for the experience theorist. (And again, part of why it is so hard is that it is far from clear that there is a unique correct answer.)

But the objective list theorist has a challenge that the experience theorist does not have: how do we weigh up the various goods involved? Let's think about a very simple list - say the only things on the list are friendship and beauty. Now in some cases, saying which of two outcomes is better will be easy. If outcome $A$ will produce improve your friendship, and let you experience beautiful things, more than outcome $B$ will, then $A$ is better than $B$. But not all choices are like that. What if you are faced with a choice between seeing a beautiful art exhibit, that is closing today, or keeping a promise to meet your friend for lunch? Which choice will maximise your welfare? The art gallery will do better from a beauty standpoint, while the lunch will do better from a friendship standpoint. We need to know something more to know how this tradeoff will be made.

There are actually three related objections here. One is that the theory is incomplete unless there is some way to weigh up the various things on the list, and the list itself does not produce the means to do the weighting. A second is that it isn't obvious that there is a unique way to weigh up the things on the list. Perhaps one person is made better off by focussing on friendship and the expense of beauty, and for another person it goes the other way. So perhaps there is no natural weighing consistent with the spirit behind the objective list theories that works in all contexts. Finally, it isn't obvious that there is a fact of the matter in many cases, leaving us with many choices where there is no fact of the matter about which will produce more utility. But that will be a problem for creating a numerical measure of value that can be plugged into expected utility calculations.

Let's sum up. There are really two core worries about objective list theories. These are:
\begin{itemize*}
\item Different things are good for different people
\item There's no natural way to produce a utility measure out of the goodness of each `component' of welfare
\end{itemize*}
Next time we'll look at desire based theories of utility, which are the standard in decision theory and in economics.