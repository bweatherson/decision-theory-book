\chapter{Group Decisions}

So far, we've been looking at the way that an individual may make a decision. In practice, we are just as often concerned with group decisions as with individual decisions. These range from relatively trivial concerns (e.g. Which movie shall we see tonight?) to some of the most important decisions we collectively make (e.g. Who shall be the next President?). So methods for grouping individual judgments into a group decision seem important.

Unfortunately, it turns out that there are several challenges facing any attempt to merge preferences into a single decision. In this chapter, we'll look at various approaches that different groups take to form decisions, and how these different methods may lead to different results. The different methods have different strengths and, importantly, different weaknesses. We might hope that there would be a method with none of these weaknesses. Unfortunately, this turns out to be impossible. 

One of the most important results in modern decision theory is the Arrow Impossibility Theorem, named after the economist Kenneth Arrow who discovered it. The Arrow Impossibility Theorem says that there is no method for making group decisions that satisfies a certain, relatively small, list of desiderata. The next chapter will set out the theorem, and explore a little what those constraints are.

Finally, we'll look a bit at real world voting systems, and their different strengths and weaknesses. Different democracies use quite different voting systems to determine the winner of an election. (Indeed, within the United States there is an interesting range of systems used.) And some theorists have promoted the use of yet other systems than are currently used. Choosing a voting system is not quite like choosing a method for making a group decision. For the next two chapters, when we're looking at ways to aggregate individual preferences into a group decision, we'll assume that we have clear access to the preferences of individual agents. A voting system is not meant to tally preferences into a decision, it is meant to tally votes. And voters may have reasons (some induced by the system itself) for voting in ways other than their preferences. For instance, many voters in American presidential elections vote for their preferred candidate of the two major candidates, rather than `waste' their vote on a third party candidate.

For now we'll put those problems to one side, and assume that members of the group express themselves honestly when voting. Still, it turns out there are complications that arise for even relatively simple decisions.

\section{Making a Decision}
Seven friends, who we'll imaginatively name $F_1, F_2, ..., F_7$ are trying to decide which restaurant to go to. They have four options, which we'll also imaginatively name $R_1, R_2, R_3, R_4$. The first thing they do is ask which restaurant each person prefers. The results are as follows.

\begin{itemize*}
\item $F_1, F_2$ and $F_3$ all vote for $R_1$, so it gets 3 votes
\item $F_4$ and $F_5$ both vote for $R_2$, so it gets 2 votes
\item $F_6$ votes for $R_3$, so it gets 1 vote
\item $F_7$ votes for $R_4$, so it gets 1 vote
\end{itemize*}

It looks like $R_1$ should be the choice then. It, after all, has the most votes. It has a `plurality' of the votes - that is, it has the most votes. In most American elections, the candidate with a plurality wins. This is sometimes known as plurality voting, or (for unclear reasons) first-past-the-post or winner-take-all. The obvious advantage of such a system is that it is easy enough to implement.

But it isn't clear that it is the ideal system to use. Only 3 of the 7 friends wanted to go to $R_1$. Possibly the other friends are all strongly opposed to this particular restaurant. It seems unhappy to choose a restaurant that a majority is strongly opposed to, especially if this is avoidable.

So the second thing the friends do is hold a `runoff' election. This is the method used for voting in some U.S. states (most prominently in Georgia and Louisiana) and many European countries. The idea is that if no candidate (or in this case no restaurant) gets a majority of the vote, then there is a second vote, held just between the top two vote getters. Since $R_1$ and $R_2$ were the top vote getters, the choice will just be between those two. When this vote is held the results are as follows.

\begin{itemize*}
\item $F_1, F_2$ and $F_3$ all vote for $R_1$, so it gets 3 votes
\item $F_4, F_5, F_6$ and $F_7$ all vote for $R_2$, so it gets 4 votes
\end{itemize*}

This is sometimes called `runoff' voting, for the natural reason that there is a runoff. Now we've at least arrived at a result that the majority may not have as their first choice, but which a majority are at least happy to vote for.

But both of these voting systems seem to put a lot of weight on the various friends' first preferences, and less weight on how they rank options that aren't optimal for them. There are a couple of notable systems that allow for these later preferences to count. For instance, here is how the polls in American college sports work. A number of voters rank the best teams from 1 to $n$, for some salient $n$ in the relevant sport. Each team then gets a number of points per ballot, depending on where it is ranked, with $n$ points for being ranked first, $n-1$ points for being ranked second, $n-2$ points for being ranked third, and so on down to 1 point for being ranked $n$'th. The teams' overall ranking is then determined by who has the most points.

In the college sport polls, the voters don't rank every team, only the top $n$, but we can imagine doing just that. So let's have each of our friends rank the restaurants in order, and we'll give 4 points to each restaurant that is ranked first, 3 to each second place, etc. The points that each friend awards are given by the following table.

\starttab{r | c c c c c c c c}
 & $F_1$ & $F_2$ & $F_3$ & $F_4$ & $F_5$ & $F_6$ & $F_7$ & Total \\ \hline
$R_1$ & 4 & 4 & 4 & 1 & 1 & 1 & 1 & 16 \\
$R_2$ & 1 & 3 & 3 & 4 & 4 & 2 & 2 & 19 \\
$R_3$ & 3 & 2 & 2 & 3 & 3 & 4 & 3 & 20 \\
$R_4$ & 2 & 1 & 1 & 2 & 2 & 3 & 4 & 15
\stoptab Now we have yet a different choice. By this method, $R_3$ comes out as the best option. This voting method is sometimes called the Borda count. The nice advantage of it is that it lets all preferences, not just first preferences, count. Note that previously we didn't look at all at the preferences of the first three friends, beside noting that $R_1$ is their first choice. Note also that $R_3$ is no one's least favourite option, and is many people's second best choice. These seem to make it a decent choice for the group, and it is these facts that the Borda count is picking up on.

But there is something odd about the Borda count. Sometimes when we prefer one restaurant to another, we prefer it by just a little. Other times, the first is exactly what we want, and the second is, by our lights, terrible. The Borda count tries to approximately measure this - if $X$ strongly prefers $A$ to $B$, then often there will be many choices between $A$ and $B$, so $A$ will get many more points on $X$'s ballot. But this is not necessary. It is possible to have a strong preference for $A$ over $B$ without there being any live option that is `between' them.  In any case, why try to come up with some proxy for strength of preference when we can measure it directly?

That's what happens if we use `range voting'. Under this method, we get each voter to give each option a score, say a number between 0 and 10, and then add up all the scores. This is, approximately, what's used in various sporting competitions that involve judges, such as gymnastics or diving. In those sports there is often some provision for eliminating the extreme scores, but we won't be borrowing that feature of the system. Instead, we'll just get each friend to give each restaurant a score out of 10, and add up the scores. Here is how the numbers fall out.

\starttab{r | c c c c c c c c}
 & $F_1$ & $F_2$ & $F_3$ & $F_4$ & $F_5$ & $F_6$ & $F_7$ & Total \\ \hline
$R_1$ & 10 & 10 & 10 & 5 & 5 & 5 & 0 & 45 \\
$R_2$ & 7 & 9 & 9 & 10 & 10 & 7 & 1 & 53\\
$R_3$ & 9 & 8 & 8 & 9 & 9 & 10 & 2 & 55 \\
$R_4$ &8 & 7 & 7 & 8 & 8 & 9 & 10 & 57
\stoptab Now $R_4$ is the choice! But note that the friends' individual preferences have not changed throughout. The way each friend would have voted in the previous `elections' is entirely determined by their scores as given in this table. But using four different methods for aggregating preferences, we ended up with four different decisions for where to go for dinner.

I've been assuming so far that the friends are accurately expressing their opinions. If the votes came in just like this though, some of them might wonder whether this is really the case. After all, $F_7$ seems to have had an outsized effect on the overall result here. We'll come back to this when looking at options for voting systems.

\section{Desiderata for Preference Aggregation Mechanisms}
None of the four methods we used so far are obviously crazy. But they lead to four different results. Which of these, if any, is the correct result? Put another way, what is the ideal method for aggregating preferences? One natural way to answer this question is to think about some desirable features of aggregation methods. We'll then look at which systems have the most such features, or ideally have all of them.

One feature we'd like is that each option has a chance of being chosen. It would be a very bad preference aggregation method that didn't give any possibility to, say, $R_3$ being chosen.

More strongly, it would be bad if the aggregation method chose an option $X$ when there was another option $Y$ that everyone preferred to $X$. Using some terminology from the game theory notes, we can express this constraint by saying our method should never choose a Pareto inferior option. Call this the \textbf{Pareto condition}.

We might try for an even stronger constraint. Some of the time, not always but some of the time, there will be an option $C$ such than a majority of voters prefers $C$ to $X$, for every alternative $X$. That is, in a two-way match-up between $C$ and any other option $X$, $C$ will get more votes. Such an option is sometimes called a Condorcet option, after Marie Jean Antoine Nicolas Caritat, the Marquis de Condorcet, who discussed such options. The \textbf{Condorcet condition} on aggregation methods is that a Condorcet option always comes first, if such an option exists.

Moving away from these comparative norms, we might also want our preference aggregation system to be fair to everyone. A method that said $F_2$ is the dictator, and $F_2$'s preferences are the group's preferences, would deliver a clear answer, but does not seem to be particularly fair to the group. There should be \textbf{no dictators}; for any person, it is possible that the group's decision does not match up with their preference.

More generally than that, we might restrict attention to preference aggregation systems that don't pay attention to \textit{who} has various preferences, just to \textit{what} preferences people have. Here's one way of stating this formally. Assume that two members of the group, $v_1$ and $v_2$, swap preferences, so $v_1$'s new preference ordering is $v_2$'s old preference ordering and vice versa. This shouldn't change what the group's decision is, since from a group level, nothing has changed. Call this the \textbf{symmetry} condition.

Finally, we might want to impose a condition that we said is a condition we imposed on independent agents: the \textbf{irrelevance of independent alternatives}. If the group would choose $A$ when the options are $A$ and $B$, then they wouldn't choose $B$ out of any larger set of options that also include $A$. More generally, adding options can change the group's choice, but only to one of the new options.

\section{Assessing Plurality Voting}
It is perhaps a little disturbing to think how few of those conditions are met by plurality voting, which is how Presidents of the USA are elected. (More precisely, it is how the electoral votes in each state are distributed. The President is chosen by a more complicated mechanism.) Plurality voting clearly satisfies the \textbf{Pareto condition}. If everyone prefers $A$ to $B$, then $B$ will get no votes, and so won't win. So far so good. And since any one person might be the only person who votes for their preferred candidate, and since other candidates might get more than one vote, no one person can dictate who wins. So it satisfies \textbf{no dictators}. Finally, since the system only looks at votes, and not at who cast them, it satisfies \textbf{symmetry}.

But it does not satisfy the \textbf{Condorcet condition}. Consider an election with three candidates. $A$ gets 40\% of the vote, $B$ gets 35\% of the vote, and $C$ gets 25\% of the vote. $A$ wins, and $C$ doesn't even finish second. But assume also that everyone who didn't vote for $C$ has her as their second preference after either $A$ or $B$. Something like this may happen if, for instance, $C$ is an independent moderate, and $A$ and $B$ are doctrinaire candidates from the major parties. Then 60\% prefer $C$ to $A$, and 65\% prefer $C$ to $B$. So $C$ is a Condorcet candidate, yet is not elected.

A similar example shows that the system does not satisfy the \textbf{irrelevance of independent alternatives} condition. If $B$ was not running, then presumably $A$ would still have 40\% of the vote, while $C$ would have 60\% of the vote, and would win. One thing you might want to think about is how many elections in recent times would have had the outcome changed by eliminating (or adding) unsuccessful candidates in this way. Or, for that matter, how the upcoming election would be changed by the presence or absence of a candidate who would likely not win (e.g., Michael Bloomberg, or a mainstream Republican running against Donald Trump).

